\chapter{简介}

\section{为什么}

“为什么用\LaTeX{}{}?”

当把\LaTeX{}{}介绍给一个人的时候,这是面对的第一个问题。
当回答“它很好用时”,又会得到第二个问题:
“Word不好用吗?”答案当然是肯定的,
Microsoft Word\index{Word}是这个世界上目前最流行的,最易用的文字处理软件之一。

但是,
我这里要做一个转折了,如果是做过比较大的文档的,几十页以上,有分章,
分节,或者还要有页眉页脚页码目录以及封面这些东西,好吧,大部分人做过唯一的这种文档,那就是毕业论文了。
给毕业论文排版,几乎是大部分人的一场梦魇。如果性格里再带一点点追求完美的
念头,追求一点点排版的质量,那给自己的毕业论文排版,那一定会留下一生深刻的印象!

页码格式不对,页码编号不对,不同章节标题字体不一致,小标号缩进老对不齐,
数字有时候是Times Roman字体,
有的时候又成了宋体,有的段落前有一块空白,有的段落间距又太小。增加了一张图,
它后面图的标号又得重新来编一次,还有图的引用处也不要忘记……增加了一段文字,发现排好
的图又跑到下面一页去了。。。。。这一页留下了好大一片空白。下一页的表格又被它推
成了分布在两页。参考文献引用增加或者删除了一个,那就得找出全文要改动的所有引用处!!
什么,你是高级用户,会使用交叉引用,好吧,那有时候打出来的文档突然出现“错误!找不到引用--”
这样的字样,你是什么心情?
生成目录字体不一样,有的标题起的名太长,把页码挤到下一行去了,一个个改好了,一全更新又完了。
如果你的文档里还有一些公式,那又是一场战争了,有的公式字显得比正文大,有的又显得比正文小,
或者总是跟写的编号对不齐,不是偏上就是偏下。公式编辑器\index{公式编辑器}版本众多,换台电脑就只能看不能改了,
公式编辑器还经常提示已过期。blablabla……

千辛万苦,终于搞定了,看起来还算漂亮,到打印店去了,omg,word版本不对,版白排了,又乱了。
用PDF吧,不同软件生成的PDF总是跟原来的样子有点差距。做完论文,感觉是扒了一层皮。

\LaTeX{}解决了这个问题,它实际可以看成是一种写给机器的语言,把格式定义好,再填上内容,
它就会按设定好的格式把一份文档生成出来,一般是生成pdf文档,整个文档的格式首先是统一的,
不会出现不同章节的显示不一样的问题,并且,文档的格式可以是封装好的,使用的时候不需要
理会具体格式,直接可以使用,封闭格式这个功能留给会的人去做,写论文不需要关心太多的格式问题。

如果是往国外投文章,那么\LaTeX{}的应用更广泛,不少国外出版社直接提供其格式文件,
只要将文档头上的$\backslash$documentclass\{article\} 大括号里的“article”换成其相应的格式文件即可,
避开了格式调整问题。

话说回来,Word还是有优势的,所见即所得,上手容易,会鼠标键盘就会用。\LaTeX{}还是需要一些入门时间的,
如果单纯会使用这个模版,我想大约需要3个小时左右,如果会处理一些常见的程序错误以及做小的格式修改,
时间就会比较长了,大约要几天,有人引导的话会快一些。
Word在做小文档,比如就几页的文档上的优势\LaTeX{}{}是无法与之比拟的,做起来是很快,比如通知啦,传单啦。
但Word存在的意义应该不是就为了那几张小广告的,几百M的体积,几百RMB的价格,当然盗版很多。在这种应用
上还不如去用一下免费的WPS,或者OpenOffice。Word支持很多特性,支持宏,支持Visual Basic……,但是,它们
又太难了,不是随便一个人就会去有兴趣学它们的,学了很少有机会用到,屠龙之技。

\LaTeX{}{}还有两个优势是Word所没有的:
\begin{enumerate}
\item{文件体积小}

使用\LaTeX{}{},所要编辑的文件以“tex”为扩展名,如果用到参考文献,可能还需要扩展名为“bib”的参考文献数据库
文件,此外,还可能有的就是文档中要插入的图片文件了。
“tex”和“bib”文件都是纯文本文件,如果愿意,可以用记事本来编辑,按照一定的格式书写,格式也是很简单的,
看到了就会使用。
而Word文件是Microsoft自己定义的二进制文件,只有用Word软件或者其它兼容的软件打开,因为是Microsoft自己的二进制格式,
因此,体积会比较大,当然文件集成度很好,只有一个文件。
相比较之下\LaTeX{}{}可能有很多文件,但这个缺陷完全可以通过压缩软件打包来完成。
如果打开一个比较大的文档,机器破的话会比较慢,而且容易出一些错误导致Word意外关闭,想来各位都遇到过这种情况的吧。
文件如果发生了意外关闭,那就有可能被损坏,损坏后就有可能。。。。。打不开了,如果没有做一些备份,
那就成了“杯具”甚至“餐具”了。相较而言,\LaTeX{}{}的文件小,而且是纯文本文件,即使被损坏,
修复起来也比Word要容易得多。

\item{使写作更加专注于内容}

说实话,这一条在学会使用\LaTeX{}{}之前,我觉得是扯淡,那时的我觉得,使用Word边写边想也一样很快的。但在我学习使用\LaTeX{}{}
的过程中,我逐渐感受到,当只面对文本,不去想它下一段怎么排,什么样的格式时,思维更加连续,写起来也更快,
而且明显感觉到自己进入了一种写作的状态,这种感觉只在以前纸上写作感觉得到,专注于自己想的内容。
这一点,只有在学会使用之后,才能去体会得到。

\end{enumerate}

\section{模版简介}

该模版以CTeX社区发布的ctexbook模版为基础,在数学系2006年模版上修改而来,删除了不兼容的旧代码,
增加了一些新版本扩展包支持的高级命令,部分功能实现方式与旧版有所不同。

该模版与研究生院网站发布的2008年模版排版效果基本相似,可直接使用。

该模版非学校官方模版,该模版可能引起的问题,{\bfseries{}模版作者不承担任何责任},特此声明。

\subsection{\LaTeX{}{}简介}

\LaTeX{}不是单指一个软件,而是指一类软件,这一类软件都是以一个基本软件\TeX\index{\TeX}为基础,
制作成宏包并经\TeX{}的原作者授权后发布。
打个比方,\TeX{}就是一些方形,三角形,圆形,半圆形的积木块,
\LaTeX{}就是另一些人用这些积木块搭成小房子,小桌子,
我们再把这些做好的小房子,小桌子摆摆成为我们的积木城市群,
就是这样。

为了国际化,\TeX{}不读“太可斯”,而按原作者Knuth, Donald Ervin的说法,
应读为“太chi”\cite{LaTeXshzh}。

Kunth是一个计算机与数学家,\TeX{}是其在1977年看到自己的成果出版时印刷质量甚不满意,
于是历时5年编写了\TeX{}排版系统,这成为西文排版业的一次重大革命。
\TeX{}系统于1982年正式定型,不再做大的改进,只修正发现的错误。
1989年,\TeX{}系统做了迄今为止最大的一次改进:支持多语言\cite{LaTeXshzh}。

\TeX{}的设计思想很简单:把一张纸看成一个坐标平面,将该坐标平面上哪一点要出现的内容标记出来,
就像这样:坐标(50,50),放置一个五号宋体的字,倾体,不加粗;从坐标(80,50)到坐标(80,200),
画一条粗为2的线,黑色……一个排好的版面就是这样被一个点一个点描述地画出来,Knuth设计的
\TeX{}指令,就是这些描述指令,然后由计算机解读,并做出最终的图,就是我们看到的排版效果。
\TeX 的精度很高,它的最小尺寸是一个叫做sp的小单位。可见光的波长近似等于100 sp,
几个sp 的误差眼睛是看不出来的\cite{texbook}。

从上面不难想到,如果只用\TeX 指令进行排版的话,会比较复杂,普通人难以胜任。
作为计算机专家的Knuth很清楚这一点,他给\TeX 设计了扩展接口。
利用这些接口,可以把\TeX 的一系列命令封装起来,做成各种各样的“宏”,这些宏,就被称为\LaTeX{}。
就像上面的比喻一样,用积木做小房子,小桌子的厂家有很多,于是就有了各种各样的\LaTeX{}版本:
MiKTeX,XeTeX,TeXLive,teTeX,fpTeX等。
使用\LaTeX{}模块,使得排版成为一件容易的事,尤其是使用已制作好的模版写文章,
不需要任务基础,知道几条语句就可以排出整齐统一的版面。
使用哪个版本的\LaTeX{}没有关系,都可以使用这个模版生成论文。这个模版制作使用的LaTeX版本是MiKTeX 2.9版。

\subsection{模版内容}

按照研究生院提供的2008版模版,毕业论文一共由封面、题名页、版权声明、勘误表、
致谢、序言、摘要、图表目录、术语表、目次、正文、参考文献、符录、索引、简历、文章列表,
一共十六部分组成\footnote{其实,还有脚注没有算上,脚注是隐含的,在这里\LaTeX{}将主动替你完成脚注的插入,而无需多作分心。就像这个脚注一样。}。

本模版将图表目录分成了图片目录与表格目录两个部分,因为这样一来清楚,二来\LaTeX{}原生支持图片与表格
分开作目录,作到一起反而费事。其它部分与研究生院模版相同。

在该模版中,任何一个部分都是可自由选择有无,如勘误表,大部分论文是没有这个部分的,
不需要某部分,只要将其生成语句注释掉即可,具体将在后面章节中讲解。
使用该模版,封面、题名页、版权声明、图片目录,表格目录、目次、参考文献、索引这八个部分不需要论文
作者直接参与,只需按要求在文中作出标记,\LaTeX{}就会替你自动生成这些部分,
而且完全不必考虑条目的编号顺序等问题。


